\documentclass{article}
\title{Lab 7, Operating Systems}
\author{Jay R Bolton}

\begin{document}

\maketitle

\section{Questions}
\begin{enumerate}


\item
\textbf{Which register number is used for the stack pointer (sp) in OS/161?}: \$29
\item
\textbf{What bus/busses does OS/161 support?}: LAMEbus
\item
\textbf{What is the difference between splhigh and spl0?}: \texttt{splhigh}
sets spl to the highest value, disabling all interrupts. \texttt{spl0} sets spl
to 0 and enables all interrupts.
\item
\textbf{Why do we use typedefs like u\_int32\_t instead of simply saying "int"?}:
My ideas: u\_int32\_t is slightly shorter than ``unsigned int." It is also more
descriptive of the size of that type (exactly 32 bits).
\item
\textbf{What does splx return?}: it assigns a new spl value and then returns
the old one.
\item
\textbf{What is the highest interrupt level?}: SPL\_HIGH
\item
\textbf{How frequently are hardclock interrupts generated?}: 100 times every second.
\item
\textbf{What functions comprise the standard interface to a VFS device?}: these
functions are in vfs.h under low-level and mid-level operations.
\item
\textbf{How many characters are allowed in a volume name?}: 32
\item
\textbf{How many direct blocks does an SFS file have?}: 15
\item
\textbf{What is the standard interface to a file system (i.e., what functions
must you implement to implement a new file system)?}: These funcs are listed in
fs.h
\item
\textbf{What function puts a thread to sleep?}: \texttt{thread\_sleep}
\item
\textbf{How large are OS/161 pids?}: 32 bits
\item
\textbf{What operations can you do on a vnode?}: all the operations are listed
in vnode.h
\item
\textbf{What is the maximum path length in OS/161?}: 1024
\item
\textbf{What is the system call number for a reboot?}: 8
\item
\textbf{Where is STDIN\_FILENO defined?}: \texttt{/kern/unistd.h}
\item
\textbf{Is it OK to initialize the thread system before the scheduler? Why
(not)?)}: No, the scheduler provides the curthread (not sure on this one).
\item
\textbf{What is a zombie?} Zombies are threads/processes that have exited but
have not been fully deleted yet.
\item
\textbf{How large is the initial run queue?}: 32
\item
\textbf{What does a device name in OS/161 look like?}: \texttt{lhd0, emu0,
somevolume, null, etc}
\item
\textbf{What does a raw device name in OS/161 look like?}: \texttt{lhd0raw}
\item
\textbf{What lock protects the vnode reference count?}:
\texttt{lock\_acquire(vn->vn\_countlock)}
\item
\textbf{What device types are currently supported?}: ``block device" and
``character device." Not sure about this one.
\end{enumerate}
\section{My Procedure Used to Create a New System Call}
\begin{enumerate}
\item 
Created the file \texttt{simple\_syscall.c} in \texttt{/src/kern/userprog/}.
\item Modified kern/arch/mips/syscall.c to have a
case for sys\_helloworld(). Copied case body from
the reboot case.

\item Created kern/userprog/simple\_syscall.c with
a function called sy \_hellworld() that takes no
arguments and simply does a kprintf.

\item Included simple\_syscalls.c into the build by
modifying conf/conf.kern to include the file

\item Modified lib/libc/syscalls.S to append
SYSCALL(helloworld, 32) to the end

\item Added "int sys\_helloworld();" to
kern/include/syscall.h

\item Modified kern/include/kern/callno.h to have "\#define SYS\_helloworld 32"

\item Modified include/unistd.h to have our
prototype without the sys\_ (int helloworld();)

\item Finally, I created a test for helloworld in
testbin. When I ran it, it worked but looped
endlessly. 

\item I then added the \_exit call by defining
sys\_\_exit in userprog, which calls thread\_exit
and then following the same procedure as above
(modifying syscalls.c). 

\item With exit implemented the test call to
helloworld did not loop, and I took that to mean
that it worked.

\end{enumerate}

\end{document}

