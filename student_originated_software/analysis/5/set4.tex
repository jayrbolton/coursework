\documentclass{article}
\usepackage{amsmath}

\title{Set 4 Homework, Analysis of Algorithms}
\author{Jay R Bolton}

\addtolength{\oddsidemargin}{-.875in}
\addtolength{\evensidemargin}{-.875in}
\addtolength{\textwidth}{1.75in}
\addtolength{\topmargin}{-.875in}
\addtolength{\textheight}{1.75in}

\begin{document}
\maketitle

\begin{itemize}
\item p 194: 8.1-4, p 197: 8.2-4, p 200: 8.3-2, p 204: 8.4-2, p 206: 8.3 or 8.4
\item p 215: 9.1-1, p 219: 9.2-1, p 223: 9.3-8 , 9.3-9, p 224: 9-2 
\end{itemize}

\section*{Chapter 8}

\begin{enumerate}

\item[\textbf{8.1-4}]

To sort each $k$ sublist, we will use an efficient comparison sort ($\Omega(n\ lg\ n)$). 

\begin{align*}
& T(n) = k\Omega(n/k\ lg\ n/k) + \Theta(1) \\
& = kc(n/k\ lg\ n/k) + d \\
& = cn\ lg\ n/k + d \\
& \geq cn\ lg\ n/k \\
& \geq cn\ lg\ k \ \ \ \ \ \ \text{because k is a constant} \\
\end{align*}

Really not sure if I did that right.

\item[\textbf{8.2-4}]

First do counting sort up to line 9 ($\mathcal{O}(n + k)$) to get C. Then we get our output with:

\begin{align*}
& result := C[a + (a-b)] \\
& result := result - C[a-1]\ \ \ if\ (a-1) \geq 0
\end{align*}

Which is $\Theta(1)$. That was an interesting/challenging puzzle.

\item[\textbf{8.3-2}]

\item[\textbf{8.4-2}]

\item[\textbf{8.3}]

\item[\textbf{8.4}]

\end{enumerate}

\section*{Chapter 9}

\begin{enumerate}

\item[\textbf{9.1-1}]

\item[\textbf{9.2-1}]

\item[\textbf{9.3-8}]

\item[\textbf{9.3-9}]

\item[\textbf{9.2}]

\end{enumerate}

\end{document}
