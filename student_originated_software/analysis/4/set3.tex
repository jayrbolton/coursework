
\documentclass{article}
\usepackage{amsmath}

\title{Set 3 Homework, Analysis of Algorithms}
\author{Jay R Bolton}

\addtolength{\oddsidemargin}{-.875in}
\addtolength{\evensidemargin}{-.875in}
\addtolength{\textwidth}{1.75in}
\addtolength{\topmargin}{-.875in}
\addtolength{\textheight}{1.75in}

\begin{document}
\maketitle

\begin{itemize}
\item p 166: 6.5-6
\item p 167: 6-1,6-2
\item p 178: 7.2-1, 7.2-5
\item p 180: 7.3-1
\item p 284: 7.4-2
\item p 185: 7-2, 7-4
\end{itemize}


\section*{Chapter 6}

\begin{enumerate}

\item[\textbf{6.5-6}]

Do `exchange' in `Heap-Increase-Key' with one assignment.

The original:

\begin{align*}
& HeapIncreaseKey(A, i, key): \\
& \ \ if\ key < A[i] \\
& \ \ \ \ error \ \text{``new key is smaller than current key''} \\
& \ \ A[i] = key \\
& \ \ while\ i > 1\ and\ A[Parent(i)] < A[i] \\
& \ \ \ \ exchange\ A[i]\ with\ A[Parent(i)] \\
& \ \ I = Parent(i)
\end{align*}

With three assignments:

\begin{align*}
& HeapIncreaseKey(A, i, key): \\
& \ \ if\ key < A[i] \\
& \ \ \ \ error \ \text{``new key is smaller than current key''} \\
& \ \ A[i] = key \\
& \ \ while\ i > 1\ and\ A[Parent(i)] < A[i] \\
& \ \ \ \ tmp = A[i] \\
& \ \ \ \ A[i] = A[Parent(i)] \\
& \ \ \ \ A[Parent(i)] = tmp \\
& \ \ \ \ i = Parent(i)
\end{align*}

With one assignment:

\begin{align*}
& HeapIncreaseKey(A, i, key): \\
& \ \ if\ key < A[i] \\
& \ \ \ \ error \ \text{``new key is smaller than current key''} \\
& \ \ while\ i > 1\ and\ A[Parent(i)] < key \\
& \ \ \ \ A[i] = A[Parent(i)] \\
& \ \ \ \ i = Parent(i) \\
& \ \ A[i] = key \\
\end{align*}

That was a real fun little puzzle.

\item[\textbf{6-1}]

\item[\textbf{6-2}]

\end{enumerate}

\section*{Chapter 7}

\begin{enumerate}

\item

\end{enumerate}

\end{document}
