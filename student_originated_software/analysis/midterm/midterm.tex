\documentclass{article}
\usepackage{amsmath}

\title{Analysis of Algorithms Midterm}
\author{Jay R Bolton}

\addtolength{\oddsidemargin}{-.875in}
\addtolength{\evensidemargin}{-.875in}
\addtolength{\textwidth}{1.75in}
\addtolength{\topmargin}{-.875in}
\addtolength{\textheight}{1.75in}

\begin{document}
\maketitle

\begin{enumerate}

\item[\textbf{1}]

	\begin{enumerate}

		\item[\textbf{\emph{{(a)}}}]

		\begin{align*}
			& T(n) = 3T(n/4) + n\ log\ n \\
			& a = 3, b = 4, f(n) = n\ log\ n \\
			& f(n) = \Omega(n^{log_4 3 + e})\ \ \text{ case 3} \\
			& (n\ log\ n) / (n^{log_4 3}) \ \ \text{ (it is polynomially larger)}\\
			& \text{Regularity condition:} \\
			& 3(n/4)log(n/4) \leq c \cdot n\ log\ n \\
			& n(3/4)(log\ n - log\ 4) \leq c \cdot n\ log\ n \\
			& n(3/4)(log\ n - 2) \leq c \cdot n\ log\ n \\
			& n(3/4)(log\ n)-(3/2) \leq c \cdot n\ log\ n \\
			& (3/4) \cdot n(log\ n)-(3/2) \leq 1 \cdot n\ log\ n \\
			& \text{(Passes)} \\\
			& T(n) = \Theta(n\ log\ n)
		\end{align*}

		\item[\textbf{\emph{{(b)}}}]

		\begin{align*}
			& T(n) = 3T(n/3) + n/3 \\
			& a = 3, b = 3, f(n) = n/3 \\
			& f(n) = \Theta(n^{log_3 3})\ \ \text{ case 2} \\
			& T(n) = \Theta(n)
		\end{align*}

	\end{enumerate}

\item[\textbf{2}]

	\begin{enumerate}

		\item[\textbf{\emph{{(a)}}}]

		$T(n) = 3T(n/4) + n\ log\ n$

		\begin{verbatim}
                 n log n
          /           |         \
(n/4)log(n/4)  (n/4)log(n/4)  (n/4)log(n/4)
...                  ...          ...
    \end{verbatim}

		Height of tree is $h = log_4 n$, and width of leaves is $n^{log_4 3}$

		$$
		T(n) = \Theta(n^{log_4 3}) + \sum_{i=1}^h 3^i((n / 4^i)\ log\ (n / 4^i))
		$$

		$T(n) = 3T(n/3) + n/3$

		\begin{verbatim}
                     n/3
          /           |         \
     (n/3)/3        (n/3)/3     (n/3)/3
...                  ...          ...
    \end{verbatim}

		Height of tree is $h = log_3 n$, and width of leaves is $n^{log_3 3}$

		$$
		T(n) = \Theta(n) + \sum_{i=1}^h (3^i ((n/3^i)/3))
		$$

		\item[\textbf{\emph{{(b)}}}]

		\begin{align*}
		& (b) \\
		& T(n) = 3T(n/4) + n\ log\ n \\
		&= \Theta(n^{log_4 3}) + \sum_{i=1}^h 3^i((n / 4^i)\ log\ (n / 4^i)) \\
		&= \Theta(n^{log_4 3}) + \sum_{i=1}^h n\cdot 3^i/4^i \cdot log\ (n / 4^i)) \\
		&= \Theta(n^{log_4 3}) + \sum_{i=1}^h n\cdot log (n/4^i) \cdot 3^i/4^i \\
		&= \Theta(n^{log_4 3}) + \sum_{i=1}^h (n\ log\ n - n\ log\ 4^i) \cdot 3^i/4^i \\
		&= \Theta(n^{log_4 3}) + \sum_{i=1}^h 3^i/4^i \cdot n\ log\ n - 3^i / 4^i \cdot n\ log\ 4^i \\
		&= \Theta(n^{log_4 3}) + \sum_{i=1}^h 3^i/4^i \cdot n\ log\ n - \sum_{i=1}^h 3^i / 4^i \cdot n\ log\ 4^i \\
		&= \Theta(n^{log_4 3}) + n\ log\ n \sum_{i=1}^h 3^i/4^i - \sum_{i=1}^h 3^i / 4^i \cdot n\ log\ 4^i \\
		&= \Theta(n^{log_4 3}) + n\ log\ n \sum_{i=1}^h (3/4)^i - n \sum_{i=1}^h (3/4)^i \cdot log\ 4^i \\
		&= \Theta(n^{log_4 3}) + n\ log\ n \sum_{i=1}^h (3/4)^i - n \sum_{i=1}^h (3/4)^i \cdot log\ 4^i \\
		&< \Theta(n^{log_4 3}) + n\ log\ n \sum_{i=1}^{\infty} (3/4)^i - n \sum_{i=1}^h (3/4)^i \cdot log\ 4^i \\
		&= \Theta(n^{log_4 3}) + n\ log\ n \left( \dfrac{1}{1-(3/4)}\right) - n\sum_{i=1}^h (3/4)^i \cdot log\ 4^i \\
		&= \Theta(n^{log_4 3}) + n\ log\ n \cdot 4 - n\sum_{i=1}^h (3/4)^i \cdot log\ 4^i \\
		&= \Theta(n^{log_4 3}) + \Theta(n\ log\ n) - n\sum_{i=1}^h (3/4)^i \cdot log\ 4^i \\
		&= \Theta(n^{log_4 3}) + \Theta(n\ log\ n) - n\sum_{i=1}^h \dfrac{3^i\ log\ 4^i}{4^i} \\
		&= \Theta(n^{log_4 3}) + \Theta(n\ log\ n) - n\sum_{i=1}^h \left(\dfrac{3\ log\ 4}{4}\right)^i \\
		&< \Theta(n^{log_4 3}) + \Theta(n\ log\ n) - n\sum_{i=1}^{\infty} \left(\dfrac{3\ log\ 4}{4}\right)^i \displaybreak[3]\\
		&= \Theta(n^{log_4 3}) + \Theta(n\ log\ n) - n\left(\dfrac{1}{1 - (3\ log\ 4/4)}\right) \\
		&\approx \Theta(n^{log_4 3}) + \Theta(n\ log\ n) - n \cdot 0.15 \\
		&= \Theta(n^{log_4 3}) + \Theta(n\ log\ n) - \Theta(n) \\
		&= \Theta(n\ log\ n)\\
		\end{align*}

		So the final guess is actually $\mathcal{O}(n\ log\ n)$ because I had to
		use A.6 twice.

		\begin{align*}
		& (b) \\
		& T(n) = 3T(n/3) + n/3 \\
		& = \Theta(n) + \sum_{i=1}^h (3^i (\dfrac{n/3^i}{3})) \\
		& = \Theta(n) + \sum_{i=1}^h \dfrac{(3^i n)/3^i}{3} \\
		& = \Theta(n) + \sum_{i=1}^h \dfrac{n}{3} \\
		& = \Theta(n) + h/3 \cdot n \\
		& = \Theta(n) + \Theta(n) \\
		& = \Theta(n)
		\end{align*}

		\item[\textbf{\emph{{(c)}}}]

		For (a), where $T(n) = \Theta(n\ log\ n)$:

		Inductive hypothesis: $T(n) \leq c \cdot n\ log\ n$ for $\mathcal{O}$ and $T(n) \geq c \cdot n\ log\ n$ for $\Omega$.

		Induction for $\mathcal{O}$:

		\begin{align*}
		& T(n) \leq 3c(n/4)\ log\ (n/4) + n\ log\ n \\
		&  = (3/4)cn(log\ n - log\ 4) + n\ log\ n \\
		&  = 3/4\cdot cn\ log\ n - 3/4\cdot cn\ log\ 4 + n\ log\ n \\
		&  \leq cn\ log\ n + n\ log\ n \\
		&  = (c+1)n\ log\ n \\
		&  \leq cn\ log\ n \\
		\end{align*}

		Induction for $\Theta$:

		\begin{align*}
		& T(n) \leq 3c(n/4)\ log\ (n/4) + n\ log\ n \\
		&  = (3/4)cn(log\ n - log\ 4) + n\ log\ n \\
		&  = 3/4\cdot cn\ log\ n - 3/4\cdot cn\ log\ 4 + n\ log\ n \\
		&  \geq 3/4\cdot cn\ log\ n - 3/4\cdot cn\ log\ n + n\ log\ n \\
		&  = cn\ log\ n \\
		\end{align*}

		For (b), where $T(n) = \Theta(n)$:

		Inductive hypothesis: $T(n) \leq c \cdot n$ for $\mathcal{O}$ and $T(n) \geq c \cdot n$ for $\Omega$.

		Induction for $\mathcal{O}$:

		\begin{align*}
		& T(n) = 3(cn/3) + n/3 \\
		& = cn + n/3 \\
		& = 4/3cn \\
		& \leq cn \\
		\end{align*}

		Induction for $\Theta$

		\begin{align*}
		& T(n) = 3(cn/3) + n/3 \\
		& = cn + n/3 \\
		& = 4/3cn \\
		& \geq cn \\
		\end{align*}

		\item[\textbf{\emph{{(d)}}}]

		Yay!

	\end{enumerate}

\item[\textbf{3}]

	\begin{enumerate}

		\item[\textbf{\emph{{(a)}}}]

		\[ \begin{matrix}
		2 & 3 & 4 & 5\\
		8 & 9 & 12 & \infty \\
		14 & \infty & \infty & \infty  \\
		16 & \infty & \infty & \infty 
		\end{matrix} \]

		\item[\textbf{\emph{{(b)}}}]

		Y[1,1] will be the least element in the matrix (least of the least of the
		columns and least of the least of the rows). If it is a singleton tableau,
		then Y[1,1] is the only populated cell. If Y[1,1] is infinity/null, then
		there is no least element and thus no elements.

		If Y[1,1] contains a non-null element then that means we have a least
		element, so m and n are 1 and our tableau is non-empty.

		\item[\textbf{\emph{{(c)}}}]


		\item[\textbf{\emph{{(d)}}}]

	\end{enumerate}

\end{enumerate}

\end{document}
