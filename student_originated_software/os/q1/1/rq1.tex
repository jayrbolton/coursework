\documentclass{article}
\title{Review Problems 1, Operating Systems}
\author{Jay R Bolton}

\begin{document}
\maketitle

\begin{enumerate}
\item[1.1]  
  \begin{enumerate}
    \item
     Processor: reads memory, executes a sequence of stored
     instructions, and writes to memory.
    \item
     Memory: stores data.
    \item
     IO Module: transfers data from internal memory to external
     memory.
    \item
     System bus: transfers data among the components of the
     processor.
  \end{enumerate}

\item[1.2]
  \begin{enumerate}
    \item
     User-visible: registers accessible by the programmer. Stores
     data and addresses of data.
    \item
     Control and status: contains either the last instruction or the
     address of the next instruction. Not controlled by programmer.
  \end{enumerate}

\item[1.3]
 \begin{enumerate}
  \item
   Processor-memory: read/write of internal memory.
  \item
   Processor-IO: read/write of external memory.
  \item
   Data processing: arithmetic or logic operations.
  \item
   Control: alteration of instruction sequence.
 \end{enumerate}

\item[1.4]
 An interrupt is a request to the processor that causes it to
 initiate an IO action and let it run in the background as the
 processor continues to work on other things. 
 
\item[1.6]
 The ratio of capacity to access time. The lower the level, the
 less capacity and higher access time. 

\item[1.7]
 The cache is smaller, faster memory placed between the CPU and main
 memory to speed up processing at a reduced cost.

 \item[1.8]
  

\item[1.10]




\end{enumerate}

\end{document}

